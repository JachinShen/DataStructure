\documentclass[UTF-8, 12pt]{ctexart}
\setmainfont{Ubuntu}
\setCJKmainfont{STXihei}
\usepackage{fancyhdr}
\usepackage{graphicx}
\title{第十一周实验报告}
\author{沈家成}
\date{\today}
\pagestyle{fancy}
\lhead{第十一周}
\chead{}
\rhead{\today}
\lfoot{}
\cfoot{\thepage}
\rfoot{}
\renewcommand{\headrulewidth}{0.4pt}
\renewcommand{\headwidth}{\textwidth}
\renewcommand{\footrulewidth}{0pt}

\begin{document}
\maketitle
\section{1602 Merge Sorted Array}
    \subsection{数列归并}
    \paragraph{}
    数列归并就是将两个有序数列组合成一个有序数列,算法思路就是每次挑两个数列头小的元素,组成新的数列。用队列实现即可。需要注意的就是,一个数列为空的时候,要单独处理。

    \subsection{数列为空时}
    \paragraph{}
    数列为空时,直接将另一个数组接上即可。
	    
\section{3002 Ruko Sort}
	\subsection{问题本质}
    \paragraph{}
    问题的本质就是去重排序,方法很多。考虑到数据量很小,选择用堆排序。
	\subsection{流程}
    \paragraph{}
    先查找这个元素是否已经存在,从而达到去重的目的。不存在就插入优先级队列(极小堆),全部插入完成后,就完成了去重,并准备好了排序。最后再出队,就完成了排序。

\section{1069 二哥的硬币}
    \subsection{主要难点}
    \paragraph{}
    这是一个动态规划的问题,主要难点在于理清递推关系,从而分而治之,降低难度。

    \subsection{初步思路}
    \paragraph{}
    设计一个二维数组dp[i][j],表示前i种硬币能否配成j,这样,递推关系就可以表示成:
    \[ dp[i][j] = dp[i-1][j-kA_i] \ (\exists k, 0\le k \le m) \]

    \paragraph{}
    这样,在$n = 2, m = 5, A_1 = 1, A_2 = 4, C_1 = 2, C_2 = 1$的情况下,产生的动态规划二维数组如下:
    \begin{equation}
        \begin{array}{ccccccc}
            i/j & 0 & 1 & 2 & 3 & 4 & 5 \\
            0   & 1 & 0 & 0 & 0 & 0 & 0 \\
            1   & 1 & 1 & 1 & 0 & 0 & 0 \\
            2   & 1 & 1 & 1 & 0 & 1 & 1 
        \end{array}
    \end{equation}

    \paragraph{}
    统计最后一行1的个数,就可以得到答案了。
    \paragraph{}
    但是,这种方法的时间复杂度为$O(m\sum_iC_i)$,经过OnlineJudge 的测试,超时,无法通过。

    \subsection{算法改进}
    \paragraph{}
    反思之前的算法,每次只得出一个布尔值,损失了很多有用的信息,所以效率不高,想要优化,就要充分利用信息。

    \paragraph{}
    重新发掘dp的用处,将dp[i][j]定义为前i 种硬币组成j 时,第i 种硬币还剩下多少,这样就能更精准地应对各种情况。

    \paragraph{}
    这样,递推关系就变成了:
    \[dp[i][j]=\left\{\begin{array}{lll}
        -1              & j<A_i, or \ dp[i][j-A_i],\\
        C_i             & dp[i-1][j] \ge 0, \\
        dp[i][j-A_i] - 1& else.
    \end{array}\right.\]
    
    \paragraph{}
    -1表示前i种硬币不能组成j,有两种情况。一是面值比j还大,只要加了1个就超过了j;二是在配更小的数的时候,硬币已经用光了。

    \paragraph{}
    那么其他情况下,就是对于组成$j-A_i$情况下,再用掉一个硬币。

    \paragraph{}
    这样,在$n = 2, m = 5, A_1 = 1, A_2 = 4, C_1 = 2, C_2 = 1$的情况下,产生的动态规划二维数组如下:
    \begin{equation}
        \begin{array}{ccccccc}
            i/j & 0 & 1 & 2 & 3 & 4 & 5 \\
            0   & 0 & -1 & -1 & -1 & -1 & -1 \\
            1   & 2 & 1 & 0 & -1 & -1 & -1 \\
            2   & 1 & 1 & 1 & -1 & 0 & 0 
        \end{array}
    \end{equation}

    \paragraph{}
    但是这种情况下,二维数组从布尔类型变成了整型,这就意味着所需要的存储空间增加了16倍,造成了内存不够。所以还需要改进。

    \subsection{继续改进}
    \paragraph{}
    时间上的问题解决了,接下来要解决空间上的问题。再观察递推公式:
    \[dp[i][j]=\left\{\begin{array}{lll}
        -1              & j<A_i, or \ dp[i][j-A_i],\\
        C_i             & dp[i-1][j] \ge 0, \\
        dp[i][j-A_i] - 1& else.
    \end{array}\right.\]

    \paragraph{}
    值得注意的是,每次递推只和$dp[i][j-A_i],dp[i-1][j]$有关,因此摒弃二维数组,只需要保存一行的数组,这样就节省了空间。

    \section{1272 写数游戏}
    \subsection{数据结构的选择}
    \paragraph{}
    总数据量不超过100,并且有排序的需求,因此选择散列表。

    \subsection{具体实现}
    \paragraph{}
    设置一个规模为100的散列表,表示第i个数字是否存在。这样通过从大到小遍历就完成了排序,而且插入新元素对于排序并没有影响。
    \paragraph{}
    这样,每次从大到小遍历,如果商不存在则增加元素,并将标志位计true,数组规模计数变量加一。当标志位在一次遍历中都没有被置true,就说明已经没有新元素了,输出数组规模即可。
\end{document}
