\documentclass[UTF-8, 12pt]{ctexart}
\setmainfont{Ubuntu}
\setCJKmainfont{STXihei}
\usepackage{fancyhdr}
\title{第三周实验报告}
\author{沈家成}
\date{\today}
\pagestyle{fancy}
\lhead{第三周}
\chead{}
\rhead{\today}
\lfoot{}
\cfoot{\thepage}
\rfoot{}
\renewcommand{\headrulewidth}{0.4pt}
\renewcommand{\headwidth}{\textwidth}
\renewcommand{\footrulewidth}{0pt}

\begin{document}
\maketitle
\section{1205 Ackerman}
    \subsection{主要难点}
    \paragraph{}
    Ackerman 函数用递归是很容易实现的,但是要非递归化,就需要借助栈的思想。
    \subsection{解决方法}
    \paragraph{}
    在手动计算的过程中,发现每次都是要提取出最后两个参数,根据情况,返回三个、两个或者一个参数,直到最后只剩下一个参数,就是最后的答案。因此,可以借用栈先进后出的特性。
    \paragraph{}
    刚开始先把要计算的 m,n 压栈,表示当前任务是计算 Ackerman(m,n),然后弹出两个元素,表示要执行任务。根据 Ackerman 函数的定义,分解为新任务,再把新任务的参数压栈。这样循环往复,直到只有一个参数,就是最终的答案了。

\section{1206 Pascal}
    \subsection{主要难点}
    \paragraph{}
    else 是可选的,这意味着即使 if-then 后面没有 else,仍然要算作正确的。在利用栈的结构检查匹配 begin end 匹配的时候,需要忽略掉还在栈顶的 if-then。
    \subsection{解决方法}
        \subsubsection{if-then-else 配对}
        \paragraph{}
    因为 if-then 是绑定在一起的,所以先把 if 压栈,如果遇到 then,就弹出栈顶检查,如果是 if,就把 then 压栈,表示这个 if 已经配对过 then 了。之后如果碰到 else,就弹栈检查是否为 then。
        \subsubsection{begin-end 配对}
        \paragraph{}
    因为 else 是可选的,就会出现多余的 then 存在栈中,因此需要忽略 then 来进行 begin-end 配对。通过不断的弹栈,直到弹出 begin 或者 栈空才结束。如果弹出 begin 就配对成功,如果栈空,就说明没有 begin 可供配对,报错。
    \subsection{小结}
    因为 else 和 if-then 不是严格的配对,所以在检查其他配对的时候,需要忽略这些。

\section{1570 100number}
    \subsection{主要难点}
    \paragraph{}
    数据量很大,最多会有 100,000 个数字,每个数字最大会达到 2,000,000,000。这就需要降低算法的时间复杂度,否则很容易超时。
    \subsection{解决方法}
    \paragraph{}
    题目中提到了数字已被排序,可以使用时间复杂度为$O(logN)$的牛顿二分法,来查找二哥喜欢的数字在数列中的位置,进而得出比它大的数字个数。
    \subsection{小结}
    \paragraph{}
    有序的数据能帮助算法降低时间复杂度。
    
\section{4009 步步为赢}
    \subsection{主要难点}
    \paragraph{}
    在分割,重组数组的时候,容易混淆变量的含义、数字的位置。
    \subsection{解决方法}
  	手动列出一步一步的过程,再照此过程编程实现。
\end{document}
