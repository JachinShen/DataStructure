\documentclass[UTF-8, 12pt]{ctexart}
\setmainfont{Ubuntu}
\setCJKmainfont{STXihei}
\usepackage{fancyhdr}
\title{第九周实验报告}
\author{沈家成}
\date{\today}
\pagestyle{fancy}
\lhead{第九周}
\chead{}
\rhead{\today}
\lfoot{}
\cfoot{\thepage}
\rfoot{}
\renewcommand{\headrulewidth}{0.4pt}
\renewcommand{\headwidth}{\textwidth}
\renewcommand{\footrulewidth}{0pt}

\begin{document}
\maketitle
\section{1039 顺序储存二叉树}
    \subsection{需求分析}
    \paragraph{}
    \subsection{具体实现}
    \paragraph{}
    \paragraph{}
    \subsection{小结}
    \paragraph{}
	    
\section{1048 二叉树遍历}
	\subsection{注意点}
    \paragraph{}
    题目中限制了是完美二叉树,要充分利用这个条件。
    \paragraph{}
	第一行的输入是结点数,不是操作数。由于第一次操作增加了三个结点,之后都是增加两个结点,因此操作数总共是结点数减一再除以二。利用完美二叉树结点数为$2^n - 1$的特性,直接除以二所得到的整数就是操作数。
    \paragraph{}
	没有父结点的是根结点。利用这个特性,没有出现在每次操作后两个数字的结点为根结点。可以设置一个布尔数组,表示每个结点是否为根结点。每次操作中,把子结点对应的位置设成false,最后剩下的true就是根结点了。
    
\section{1221 bst}
    \subsection{代码复用技巧}
    \paragraph{}
    书上的代码只有AVL树的插入和删除一个结点的操作,而题目中有删除大量结点的操作,将其分解为一个一个删除元素就尤其重要。
    \paragraph{}
    在删除大量结点的时候,原本将程序设计成先尽可能多地删除结点,最后再一起调整平衡。造成的问题就是不平衡的情况非常复杂,很难保持平衡。而且一旦无法保持平衡,之后的基于平衡假设的操作也很可能出现运行错误。因此,复用已经设计好的删除操作是更好的解决方案。
    \paragraph{}
    每次发现需要删除的元素,只删除该元素并保持平衡,再删除下一个元素。看似有很多冗余操作,但是每次删除时都保持了平衡,免除了最后调整平衡的麻烦,其实并没有多出多少运算量,还保证了程序的稳定性。
    
    \subsection{改造技巧}
    \paragraph{}
    AVL树是忽略重复元素的,但是题目中要求记录重复元素,这就需要改造AVL树来适应需求。有两种思路,一是把重复元素也插入到树中,而是用一个结点的成员变量来记录重复次数。
    \paragraph{}
    第一种思路的好处是对插入操作只需要做少许修改,但是二叉查找树左小右大的特性就被削弱了,可能造成查找、删除操作的错误。尤其是面对题目中要求的删除区间元素,等于情况的处理就会特别麻烦,因此没有选用。
    \paragraph{}
    第二种思路的优点是保持了二叉查找树的特性,并且与原本的AVL树操作相容性好。
    \subparagraph{}
    插入操作只需要加一个插入元素与结点元素是否相等的判断,相等就将计数变量加一,直接结束即可,不相等则进入原本的流程。由于没有在物理上插入新结点,并不需要开辟新空间和调整平衡。
    \subparagraph{}
    删除操作也只需要判断要删除的元素与结点元素是否相等,相等就变量减一,如果还有剩余的就直接退出,没有以及不相等则继续原本流程。
    
    \subsection{测试技巧}
    \paragraph{}
    使用OJ评测的时候,无法重现bug,只能自己测试。但是OJ评测时往往有大量操作,自己手动不太现实。因此,需要编写测试程序。
    \paragraph{}
    为了测试程序的稳定性,可以用成千上万次的随机操作来考验它。随机数的种子应该设为固定值,这样出现问题之后,可以重现。随机数可以用来生产操作和数据,利用取余限定在一个范围内。gdb是很好的调试工具,在gdb中运行程序,出现错误时会停在错误的地方,方便在“犯罪现场”“收集证据”。
    \paragraph{}
    刚开始测试时,规模应该由小逐渐增大,每次增加半个数量级为佳。出现运行错误后,首先利用gdb查看出错的原因,常见的有访问空地址、内存泄露等,然后在原始程序中添加输出,便于定位。
    \paragraph{}
    再次运行后,就可以知道出现错误的循环次数。然后在错误前十次左右设置断点,查看有没有超出预期的数据。
    \paragraph{}
    我就是在错误前查看了数次的运行,才发现在删除大量数据的函数中,将当前结点而不是根节点传给了删除函数,导致AVL树的失衡,出现了错误。这样的bug单单看代码是很难发现,通过测试找到问题,回溯过程,才能发现这些隐藏极深的bug。

\end{document}
