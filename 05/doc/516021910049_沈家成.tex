\documentclass[UTF-8, 12pt]{ctexart}
\setmainfont{Ubuntu}
\setCJKmainfont{STXihei}
\usepackage{fancyhdr}
\title{第五周实验报告}
\author{沈家成}
\date{\today}
\pagestyle{fancy}
\lhead{第五周}
\chead{}
\rhead{\today}
\lfoot{}
\cfoot{\thepage}
\rfoot{}
\renewcommand{\headrulewidth}{0.4pt}
\renewcommand{\headwidth}{\textwidth}
\renewcommand{\footrulewidth}{0pt}

\begin{document}
\maketitle
\section{1111 二哥学二叉树}
    \subsection{问题分析}
    \paragraph{}
    这是一个递归的问题,难点在于应该往下传递哪些递归参数,才能达到递归解决的效果哦。为了理清整个过程,需要首先手工模拟这个过程。
    
    \subsection{过程分析}
    \paragraph{}
    第一步,找到前序遍历的第一个元素,作为根节点。将根节点放入顺序数组。
    \paragraph{}
    第二步,找到根节点在中序遍历中的位置,从而将中序遍历分为左右两部分。
    \paragraph{}
    第三步,找到中序遍历左边/右边部分对应的前序遍历,将当前根节点在顺序数组中的位置,左/右半中序遍历,及其前序遍历传入递归函数。
    
    \subsection{代码细节}
    \paragraph{}
    递归终止的条件:当新的前序排列只剩一个元素的时候,只要将其放入顺序数组,不用再往下递归了。
    \paragraph{}
    如何查找中序遍历元素对应的前序元素:经过观察,发现左/右半中序遍历总是在前序中成团出现,因此只要找到任意左/右半中序元素在前序遍历中的位置,就找到了整个左/右半中序遍历对应的前序遍历元素。

\section{1211 isCBT}
    \subsection{难点分析}
    \paragraph{}
    从输入的数据获得二叉树的结构不是很难,困难的在于如何判断是不是完全二叉树。完全二叉树只允许最后一层不满,并且叶子结点都要靠左。叶子结点容易分辨,但是怎么知道它处于最底层哪个位置呢?就要往上搜索,一直回溯到根结点才可以进行判断,这样就很复杂。
    \subsection{解决方法}
    \paragraph{}
    顺序数组存储二叉树的时候,如果是完全二叉树,元素中间是不会有空洞的,因此可以利用这个性质,判断是否为二叉树。
    \paragraph{}
    首先寻找根结点。通过搜寻没有成为子结点的结点来寻找根结点。然后从根结点开始遍历,将结点编号储存进顺序二叉树数组。最后判断数组中是否有空洞,从而判断
    \subsection{再思考}
    在进行数组顺序储存的时候,常常会越界。一种解决办法是算出层数,预留空间;另一种办法是每次储存时进行判断是否越界,后来进一步想,为什么会越界呢?因为不够存储。那又为什么不够存储?说明之前有空余的空间,正好就是因为不是完全二叉树。因此,只要判断储存数组时会不会越界,就可以间接判断是不是完全二叉树了。

\section{1214 traverse}
    \subsection{孩子兄弟表示法}
    \paragraph{}
    孩子兄弟表示法的前序遍历和后序遍历只需要用递归的方法,重复调用自身就可以了。然而,层次遍历因为要分层输出,就不适用于递归了,而是适用队列的方法。

    \subsection{具体实现}
    \paragraph{}
    首先输出根结点,并将根结点的孩子们依次放入队列,这样下一步的任务就是层次遍历这些孩子结点。然后,从队列读取任务,输出当前结点,并将当前结点的孩子们加入队列任务,为下一层结点输出做准备。这样一直重复,直到队列为空,就利用队列后进后出的特性,实现了队列。
    \subsection{小结}
    \paragraph{}
    根据实际情况的需求,选择数据结构,会极大地方便思路的整理和代码的实现。
\end{document}
