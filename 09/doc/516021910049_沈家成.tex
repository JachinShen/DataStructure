\documentclass[UTF-8, 12pt]{ctexart}
\setmainfont{Ubuntu}
\setCJKmainfont{STXihei}
\usepackage{fancyhdr}
\title{第九周实验报告}
\author{沈家成}
\date{\today}
\pagestyle{fancy}
\lhead{第九周}
\chead{}
\rhead{\today}
\lfoot{}
\cfoot{\thepage}
\rfoot{}
\renewcommand{\headrulewidth}{0.4pt}
\renewcommand{\headwidth}{\textwidth}
\renewcommand{\footrulewidth}{0pt}

\begin{document}
\maketitle
\section{1216 heap}
    \subsection{find 操作}
    \paragraph{}
    find操作只要求$O(n)$的时间复杂度,因此,只需要在储存二叉堆的顺序数组中遍历即可。
    \paragraph{}
    但是,find 操作的逻辑需要仔细考量。首先,要求是大于x的最小优先级,然后才是在相等的优先级里找下标最小的。这就涉及到最大值最小值,刚开始选择最后一个元素作为"擂主",但是它不一定比x 小,因此最后设置了一个布尔变量来找到第一个大于 x 的元素。因此,编写程序的时候,要注意容易被默认的前提条件。

    \subsection{decrease 操作}
    \paragraph{}
    刚开始设想利用原有的函数,先删除那个元素,再插入优先级减少后的元素,但是这样会打乱整个二叉堆的结构。因此需要把优先级减少后,再调整二叉堆的结构。
    \paragraph{}
    借鉴插入堆的操作,将优先级减少后,和父节点作比较,如果比父节点小,就和父节点交换,否则就跳出。
	    
\section{1588 导弹拦截}
	\subsection{单系统最大拦截数}
    \paragraph{}
    因为每一发导弹都不高于前一发导弹,所以导弹的高度就组成了一个递减数列。因此单系统最大拦截数,就是求最大递减子数列。而在1580中已经解决了最大递增子数列的问题,替换比较运算符即可。
    \subsection{最少拦截系统数}
    \paragraph{}
    第一发导弹发出后,只能击落比它高度低的导弹,这样,一旦有突然增大的高度,就需要另一套系统来拦截。因此,这就是求最大递增子数列的问题。因为,对于最大递增子数列中的每两个元素,在原数列中对应位置之间的元素,一定小于前一个元素和后一个元素,否则递增子数列中的元素就是那个反例了。
   
\section{4012 合并果子}
    \subsection{问题的本质}
    \paragraph{}
    每种果子都是要至少合并一次的,节省体力就是要数目多的那种果子合并的次数少,把数目看做权值,就很容易联想到我们学习过的编码问题,也是同样的,要用尽量少的编码表示出现频率高的字符。因此,可以选用霍夫曼算法解决这个问题。
    
    \subsection{具体实现}
    \paragraph{}
    这个问题可以借用霍夫曼算法的思想,但是最后只是需要输出最小体力值,并不需要得到编码,因此使用贪心法就可以了。
    \paragraph{}
    根据贪心法的思路,每次要找出最小和次小的元素,然后再把结果放回原来的数据中。容易想到用动态查找表保证次序,但其实只需要用最小化堆实现的优先级队列即可。
    \paragraph{}
    刚开始把元素都放入优先级队列。然后两次读队,就恰好得到了权值最小和次小的元素。权值相加,再插入优先级队列,就相当于完成了一次合并操作。每次操作的权值和就是消耗的体力值,累加,就得到了总体力值。
    
\end{document}
