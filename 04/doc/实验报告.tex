\documentclass[UTF-8, 12pt]{ctexart}
\setmainfont{Ubuntu}
\setCJKmainfont{STXihei}
\usepackage{fancyhdr}
\title{第四周实验报告}
\author{沈家成}
\date{\today}
\pagestyle{fancy}
\lhead{第四周}
\chead{}
\rhead{\today}
\lfoot{}
\cfoot{\thepage}
\rfoot{}
\renewcommand{\headrulewidth}{0.4pt}
\renewcommand{\headwidth}{\textwidth}
\renewcommand{\footrulewidth}{0pt}

\begin{document}
\maketitle
\section{1005 数独}
    \subsection{走过的弯路}
    \paragraph{}
    在思考如何检查 1~9 每个数字都出现的时候,想简单了,认为只要直接每行、每列、每个九宫格的数字相加,看看是不是 45 就可以了。后来想到没那么简单,可能会有一对 1-9 替换 2-8 的情况,但是这样和依然是 45。
    \subsection{解决办法}
    \paragraph{}
    定义了一个 9 位的布尔型数组,用来记录每个数字是否出现。遍历完一行、一列或一个九宫格后,就检查这个数组是不是全部为 true。
    \subsection{小结}
    \paragraph{}
    为了提高准确率,要充分发掘信息,宽松的必要条件是不够的,需要严格的必要条件,最好找到充要条件。

\section{1049 火车调度}
    \subsection{主要难点}
    \paragraph{}
    用到了栈和队列两种数据结构,整个程序很复杂。
    \subsection{解决方法}
    \paragraph{}
    模块化开发。先测试自己写的栈类和队列类,入栈、出栈、入队、出队等功能都正常运行后,再编写专门用于本题的类。
    \subsection{小结}
    在编写大型程序的时候不能着急,要及时测试模块运行是否正常。如果到了最后才一起 debug,很难确定 bug 的地点。

\section{1574 调皮的助教}
    \subsection{难点分析}
    \paragraph{}
    助教和题的关系明显是一个循环的结构。由于需要多次调用,对于出题算法的时间复杂度要求特别高。
    \paragraph{}
    最终将会出$10^{1,000,000,000}$次题,如果完全依靠计算机计数,将会是指数级的时间复杂度,很可能宕机。
    \subsection{解决方法}
    	\subsubsection{看待角度}
    	\paragraph{}
    看待问题的角度不同,算法也就不同。用循环的结构来储存数据的时候,把谁看作数据,把谁看作地址;又把谁看作动的,把谁看作静的。一般来说,地址是静的,数据是动的。那么最符合常识的看法是,把助教看作静止的地址,把题看作运动的数据,每次出题,就是每个地址的数据改变。但是这个每次都要涉及数据的大量移动,是$O(N)$的时间复杂度。
    	\paragraph{}
    但是如果换一种看法呢?如果把地址看作动的,把数据看作静止的,那么是每道题对应的助教不同,再把物理地址抽象成逻辑地址,只要知道 0 号助教出第几道题,不就知道所有其他助教出什么题了吗?所以最终,设置一个 n 个元素的数组,里面存放 0~n-1 号题,再定义一个变量表示 0 号助教在数组中的位置。每次出题,就是 0 号助教的位置加 m,通过取余限制在循环范围内。这样,每次只需要进行一次运算,时间复杂度为$O(1)$。
    	\paragraph{}
    甚至再进一步,0 号助教所处的位置,和出题号在数值上是一致的,因此都不需要设置一个数组,直接用位置就可以表示出题号了。
    	\subsubsection{递归计算}
    	\paragraph{}
    递归函数常常会消耗大量资源,但是如果使用递归的思想来进行$10^{1,000,000,000}$次循环,就可能将其降到$1,000,000,000$次循环,这是计算机能够解决的问题。
    	\paragraph{}
    $10^{1,000,000,000}$次循环,可以看作 10 次$10^{999,999,999}$循环……100 次循环,可以看作 10 次 10 循环。而每层循环的结果,可以看作助教的地址又向后移动了几位。因此,只要记录下移动的距离,将每十次移动的最终距离传给上一层,作为上一层每次移动的距离,就实现了递归操作。
    	\paragraph{}
    即使这样,十次测试中依然有一次超时。因此,可以一次循环里进行三次递归,也就是计算移动一千次后的位置,最后再根据三的余数处理各种情况。
    \subsection{小结}
    \paragraph{}
    如果一个角度看问题很复杂,换个角度看问题,可能就简单多了。
    \paragraph{}
    递归函数经常通过多次运行来得到数值,如果反着想,就可以用数值代表多次运算,从而降低时间复杂度。
\end{document}
